\documentclass[a4paper]{article}
\usepackage[utf8]{inputenc}
\usepackage{geometry}
\usepackage{amsmath}
\pdfpagewidth
\paperwidth
\pdfpageheight
\paperheight
\usepackage{booktabs}
\usepackage{graphicx}
\usepackage{subfig}
\usepackage{verbatim}
\newcommand*{\unit}[1]{\ensuremath{\mathrm{\,#1}}}
\usepackage{amsthm}
\usepackage{epsfig}
\usepackage{fancyhdr} 
\usepackage{amsmath,amssymb}
\usepackage{amscd} 
\usepackage[T1]{fontenc} 
\usepackage[utf8]{inputenc} 
\usepackage[usenames,dvipsnames]{xcolor}
\usepackage{graphicx,color,listings}
\usepackage{hologo}
\frenchspacing 
\usepackage{float}
\usepackage{geometry}
\usepackage{rotating}
\usepackage{caption}
\captionsetup{labelformat=empty, textfont=sl}
\usepackage{placeins}
\frenchspacing
\title{Esperienza Laboratorio di Fisica Medica: Spettroscopia}
\author{Simone Lossano, Lorenzo Marini, Jake Harold Pensavalle}
\begin{document}
	\maketitle
	\newpage
	\tableofcontents
	\newpage
%%%%%%%%%%%%%%%%%%%%%%%%%%%%%%%%%%%%%%
\section{Linearità, risoluzione energetica e calibrazione}
In questa sezione abbiamo presto manualità con la strumentazione acquisendo vari spettri di decadimento di varie sorgenti radioattive. Lo scopo di questa prima parte dell'esperienza è di trovare una relazione tra energia e numero del canale per poter calibrare il rivelatore. 
Si riportano i vari spettri di decadimento e la calibrazione. Come funzione di fit per la calibrazione si è utilizzata una funzione lineare con parametri riportati in tabella:

	\begin{center} 
		
		\begin{tabular}{lcccc}
			\hline
			\hline
			\textbf{Coefficiente lineare}	& \textbf{intercetta}	& \textbf{$\chi^{2}_{redux}$} & \textbf{pvalue} \\
			\hline
			\hline
			 $0.737	\pm 0.005$	& $-16.37 \pm 0.1	$			& 0.2 &0.998	\\
			
			\hline
			\hline
		\end{tabular}
		\linebreak
		
	\end{center}
	\emph{Tab.1: risultati del fit. Il valore non ottimale del chi quadro è dovuto al fatto che si sono usati pochi punti per il fit.} 
\begin{figure}[H]%
    \centering
    \subfloat[]{{\includegraphics[width=5cm]{americio_senza_fondo} }}%
    \qquad
    \subfloat[]{{\includegraphics[width=5cm]{cesio_senza_fondo} }}%
    \qquad
    \subfloat[]{{\includegraphics[width=5cm]{cobalto_senza_fondo}}}%
    \qquad
    \subfloat[]{{\includegraphics[width=5cm]{sodio_senza_fondo}}}%
    \qquad
    \subfloat[]{{\includegraphics[width=5cm]{fit_gaussiano_Americio}}}%
    \qquad
    \subfloat[]{{\includegraphics[width=5cm]{fit_gaussianocesio}}}%
    \qquad
    \subfloat[]{{\includegraphics[width=5cm]{fit_gaussiano_cobalto1} }}%
    \qquad
    \subfloat[]{{\includegraphics[width=5cm]{fit_gaussiano_cobalto_secondo_picc} }}%
    \qquad
    \subfloat[]{{\includegraphics[width=5cm]{fit_gaussiano_Na}}}%
   \caption{Grafici degli spettri eliminando opportunamente il fondo e fit gaussiani dei picchi. Del Cobalto si riportano entrambi i picchi mentre per il sodio la statistica non è sufficiente per distinguere il secondo picco con il metodo del fit gaussiano}%
    \label{fig:1}%
\end{figure}

\begin{figure}[!h]
\includegraphics[width=1\textwidth]{calibrazione}
        \caption{Calibrazione del rivelatore. Il punto corrispondente al secondo picco del sodio è stato stimato visivamente nello spettro}
        \label{fig:2}
\end{figure}





\section{Verifica di 1/$r^2$}
In questa sezione si verifica l'andamento dell'energia (dipendente dall'area sotto il fotopicco) in funzione della distanza. Si sono eliminati gli outlier corrispondenti ai punti più vicini al rivelatore. Come aspettato stando troppo vicino al rivelatore vi sono fenomeni di pile-up che rendono invalide tali acquisizioni. Come modello si è utilizzato \begin{equation}
A_{fotopicco}=\frac{C}{r^{2}}+B
\end{equation}
Con A e B parametri stimati dal fit.
\begin{figure}[H]
\includegraphics[width=1\textwidth]{inverserootlawwithpars}
        \caption{Andamento area sotto il fotopicco in funzione della distanza.}
        \label{fig:2}
\end{figure}

\section{Calcolo di $\mu$/$\rho$ di Cu e di Al}
Ponendo vari strati di materiali assorbenti, si possono stimare i $\mu$ dei materiali. La funzione di fit utilizzata è:
\begin{equation}
A_{fotopicco}=A_{0}e^{-\mu x} + B
\end{equation}
Con $A_{0}$ e B parametri liberi.

\begin{center} 
		
		\begin{tabular}{lccccc}
			\hline
			\hline
			\textbf{Materiale}\textbf{Coefficiente di assorbimento }	& \textbf{costante additiva}	& \textbf{$\chi^{2}_{redux}$} & \textbf{pvalue} \\
			\hline
			\hline
			 Cu&$0.575	\pm 0.009$	& $14068.144 \pm 78.833	$			& 1.8 &0.113	\\
			Al&$0.194 \pm 0.06$ &$10553 \pm 148.143$ & 8.213 & 0.000 \\
			\hline
			\hline
		\end{tabular}
		\linebreak
		
	\end{center}
	\emph{Tab.2: risultati del fit esponenziale. Il valore non ottimale del chi quadro è dovuto al fatto che si sono usati pochi punti per il fit, limitati dal numero di spessori utilizzati.} 
\subsection{Confronto $\frac{\mu}{\rho}$ tra dati sperimentali e tabulati}
Conoscendo la densità del materiale si può stimare il rapporto $\frac{\mu}{\rho}$:
\begin{center} 
		
		\begin{tabular}{lccc}
			\hline
			\hline
			\textbf{Materiale}\textbf{$\frac{\mu}{\rho}_{exp} $}	& \textbf{$\frac{\mu}{\rho}_{theo} $}	 \\
			\hline
			\hline
			 Cu&$0.065 \pm 0.001	$	& $0.0066$			\\
			Al&$0.072 \pm 0.003$ &$0.068$ \\
			\hline
			\hline
		\end{tabular}
		\linebreak
		
	\end{center}
	\emph{Tab.3: Confronto tra valori sperimentali e valori attesi. Possiamo notare un buon accordo tra i valori.} 

\section{Calcolo dell'efficienza intrinseca di picco in funzione dell'energia}
Utilizzando una attività al tempo zero nominale di 78 kBq e usando come tempo di riferimento il giorno 18/02/05, si possono calcolare le attività e si possono calcolare le efficienze utilizzando:
\begin{equation}
\epsilon_{int}=\epsilon_{abs}4 \frac{\pi}{\omega}
\end{equation}
Dove $\omega$ è l'accettanza geometrica stimata dalla geometria del sistema, ed $\epsilon_{abs}$=$\frac{# eventi rilevati }{# eventi totali}$ dove il numero di eventi totali si stima sapendo l'attività e i tempi di misura.

\begin{figure}[H]
\includegraphics[width=1\textwidth]{efficiency}
        \caption{Efficienza in funzione dell'energia. Le scarse efficienze per il Co e Na sono dovute alla statistica non ottimale per quei punti}
        \label{fig:2}
\end{figure}


\end{document}